\section{Проблема}
\begin{frame}
\frametitle{Проблема}
Бизнес регулярно сталкивается с задачами, которые можно эффективно решить с помощью алгоритмов машинного обучения:
\begin{itemize}
    \item классификация спам/не спам
    \item распознавание образов на видео
    \item детекция и фильтрация недопустимого или даже запрещённого контента
    \item и пр
\end{itemize}

Практика при этом показывает, что \textbf{такие задачи похожи или имеют типовое решение.}

\end{frame}

\begin{frame}
\frametitle{Типовое решение}
% Решение при этом сводится к двум вещам:
\begin{enumerate}
    \item Решение задачи машинного обучения:
    \begin{itemize}
        \item постановка задачи
        \item сбор и разметка данных
        \item выбор архитектуры
        \item обучение и валидация
    \end{itemize}
    \item Публикация модели и данных:
    \begin{itemize}
        \item в виде Google Colab ноутбука (?)
        \item в виде своей странички:
        \begin{itemize}
            \item приходится администрировать свой сервер;
            \item приходится писать код;
            \item платить за хостинг;
        \end{itemize}
    \end{itemize}
\end{enumerate}
\end{frame}

\begin{frame}
\frametitle{Аналоги}
\end{frame}
