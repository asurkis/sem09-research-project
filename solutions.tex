\section{Решения}

\begin{frame}
\frametitle{Задача \ref{itm:task:infra}: разработать клиент для инфраструктуры ОК}
\begin{block}{Взаимодействия с Airflow API}
    \begin{itemize}
        \item Получение списка моделей
        \item Запуск и остановка обучения моделей
        \item Получение информации о статусе обучения
        \item Выбор параметров модели
        \item Экспорт метрик
    \end{itemize}
\end{block}

\begin{block}{Конечный автомат состояний обучения модели}
    \begin{itemize}
        \item Непосредственно конечный автомат
        \item \textdim{Data Access Object для хранения его в БД}
    \end{itemize}
\end{block}

\begin{block}{\textdim{База данных}}
    \begin{itemize}
        \item \textdim{Схема БД для хранения информации о моделях и пользователях}
        \item \textdim{Клиентский код для взаимодействия с Cassandra}
    \end{itemize}
\end{block}
\end{frame}

\begin{frame}
\frametitle{Использованные технологии}
Airflow и Cassandra уже существуют и развёрнуты у ОК.

\bigskip

Основа сервиса --- Java + Spring Boot.
Так ОК смогут поддерживать проект после завершения моей работы над ним.
\end{frame}

\begin{frame}
\frametitle{Состояния одного процесса}
\begin{center}
    \begin{tikzpicture}[node distance=2cm,>={stealth},line width=1pt]
        \node (running)                                                        {Запущен}         ;
        \node (queued)     [ left        of = running    , node distance=3cm ] {В очереди}       ;
        \node (stopping)   [ below       of = running                        ] {Останавливается} ;
        \node (finished)   [ below left  of = stopping                       ] {Завершился}      ;
        \node (terminated) [ below right of = stopping                       ] {Упал}            ;
        \node (restarting) [ right       of = terminated , node distance=3cm ] {Перезапускается} ;

        \draw[->] (queued)     to                      (running);
        \draw[->] (queued)     to [ out=225          ] (finished);
        \draw[->] (restarting) to [ out=90  , in=0   ] (running);
        \draw[->] (restarting) to [ out=225 , in=315 ] (finished);
        \draw[->] (running)    to                      (stopping);
        \draw[->] (running)    to [ out=225          ] (finished);
        \draw[->] (running)    to [ out=315 , in=45  ] (terminated);
        \draw[->] (stopping)   to                      (finished);
        \draw[->] (stopping)   to                      (terminated);
        \draw[->] (terminated) to                      (restarting);
    \end{tikzpicture}
\end{center}
\end{frame}
